% Physics Homework Template using MEMOIR class
\documentclass[openany]{memoir}

%preamble
\usepackage{calc}
\usepackage{color}
\usepackage{graphicx}

%define title and other basic document info
%the title should reflect the style and give a foretaste of the document
%work on making a stylized title page � or title on a page as in ARTICLE class
\title{\huge \textbf{Ch 15 Recitation Problems}}
\author{J Edward Ladenburger}
\date{05 Nov 08}                    % could use \today  , but I like this date format better
%\publisher{}                            %one day I�ll need this  
%\thanks{Special thanks to God for the ability to work}        %produces a footnote to the title

\definecolor{shadecolor}{gray}{0.9}
\definecolor{ared}{rgb}{.647,.129,.149}
\renewcommand\colorchapnum{\color{ared}}
\renewcommand\colorchaptitle{\color{ared}}
\chapterstyle{bringhurst}
%one of a number of chapter styles available�this one doesn�t use the ared color
%%%%%%%%%%%%%%%%%%%%%%%%%%%%%%%%%%%%%%%%%%%%%%%%

\begin{document}
%title
\thispagestyle{empty}
%\begin{minipage}{300pt}
\begin{center}{
\begin{shaded}
\hrule \vspace{30pt}
\hspace{10pt} \thetitle  \vspace{30pt}
\newline \theauthor \hspace{30pt} \thedate  \vspace{26pt}
\hrule
\end{shaded}
}
\end{center}
%\end{minipage}
\clearpage

%\frontmatter    %use if needed �page numbers as lower case roman numerals i, ii,�

%\mainmatter
%%other declarations
\pagestyle{Ruled}                    %one of a number of possible page styles
\midsloppy                             %to minimize overfull lines

%Layout the page
%%Try this manual golden ratio layout or�           default seems better for now
%\settypeblocksize{*}{\lxvchars}{1.618}
%\setulmargins{50pt}{*}{*}
%\setlrmargins{*}{*}{1.618}
%\setheaderspaces{*}{*}{1.618}
%\semiisopage[12]
%try this predefined layout � others predefined ones are options in MEMOIR�
%this one looked best but did not work

\checkandfixthelayout          %make the layout happen and provide details in log during build

\chapter{Chapter 15 Recitation Problems}
\section{Problem 3}
\subsection{Given}
A 6m x 12m swimming pool which slopes linearly from a 1.0m depth at one end to a 3.0m depth at the other end.
\subsection{Find}
The mass of water in the pool. $M_{H2O}$
\subsection{Plan}
Simply find the volume of the pool and multiply by the density of water. $M_{H2O}=\rho \cdot V$\\
\[ V=A_{trapezoidal side}\times Width \]
\subsection{Calculations}
\begin{eqnarray*}
V &=& \frac{3+1}{2} \times 12 \times 6\\
V &=& 24 \times 6 \\
V &=& 144 m^3 \\
M_{H2O} &=& 144 m^3 \times \frac{1000 kg}{m^3} \\
M_{H2O} &=& 144000 kg
\end{eqnarray*}
\subsection{Solution}
\begin{minipage}{300pt}
\begin{center}{
\begin{shaded}
\hrule
\vspace{20pt}
The mass of water in the pool is $1.44 \times 10^5 kg$         %nicely written sentence solution goes here
\vspace{16pt}
\hrule
\end{shaded}
}
\end{center}
\end{minipage}

\section{Problem 5}
\subsection{Given}
The deepest point in the ocean is 11 km below sea level.
\subsection{Find}
The pressure in atmospheres at this depth.
\subsection{Plan}
The hydrostatic pressure at a depth, d is $P = P_o + \rho g d$. I just need take care with the units.
\subsection{Calculations}
$P = 1 atm + 1030 \frac{kg}{m^3} \times 9.8 \frac{N}{kg} \times 11000 m \times \frac{1 atm}{1.013 \times 10^5 Pa}$
\subsection{Solution}
\begin{minipage}{300pt}
\begin{center}{
\begin{shaded}
\hrule
\vspace{20pt}
The pressure at 11 km below sea level is ~1097 atmospheres.   %nicely written sentence solution goes here
\vspace{16pt}
\hrule
\end{shaded}
}
\end{center}
\end{minipage}

\section{Problem 9}
\subsection{Given}
A submarine with a 20 cm diameter window which is 8.0 cm thick.  The manufacturer says it can stand forces up to $1.0 \times 10^6 N$. The pressure inside the submarine is maintained at 1.0 atm.
\subsection{Find}
The maximum safe depth for the submarine.
\subsection{Plan}
Since $P = \frac{F}{A}$, I can just find the depth at which the pressure will reach the manufacturers maximum force for the area of the given window.  Again, paying close attention to the units. The fact that the inside of the submarine is maintained at 1.0 atm allows me to use $P_{max} = \frac{F_{max}}{A}=\rho g d$ and solve for $d$.
\subsection{Calculations}
The area of the window is,\\
\[ \pi \times (0.10 m)^2 = \frac{\pi}{100} m^2 \]\\
So the maximum pressure for this window is, \\
\[ \frac{(1.0 \times 10^6 N)}{\frac{\Pi}{100} m^2} = 3.183099 \times 10^7 Pa\]\\
The depth at which this pressure is reached is,\\
\[ \frac{3.183099 \times 10^7 Pa}{1030 \frac{kg}{m^3} \times 9.8 \frac{N}{kg}} = 3153.5 m \]
\subsection{Solution}
\begin{minipage}{300pt}
\begin{center}{
\begin{shaded}
\hrule
\vspace{20pt}
The submarine can descend to a depth of ~3150 meters.                %nicely written sentence solution goes here
\vspace{16pt}
\hrule
\end{shaded}
}
\end{center}
\end{minipage}

\end{document}
